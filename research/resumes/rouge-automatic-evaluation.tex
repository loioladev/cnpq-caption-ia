% O documento possui um erro de segurança por causa do comando -shell-escape na compilacao
\documentclass[12pt]{article}
\usepackage[utf8]{inputenc}
% Colocar documento em portugues
\usepackage[brazil]{babel}
% Incluir asmmath e outras ferramentas matemátias
\usepackage{mathtools}

% Pacote para demonstração
\usepackage{amsthm}

% Pacote para cores
\usepackage{xcolor}
%\definecolor{pastelyellow}{HTML}{fdfd96} - Definir cor

% Pacote matemático
\usepackage{amsmath}

% Pacote para imagens um ao lado da outra
\usepackage{subfig}

% Pacote e função para espaçamento 1.5
\usepackage{setspace}
\onehalfspacing%

% Pacote para códigos
\usepackage{minted}

% Para representar símbolos dos conjuntos
\usepackage{amsfonts}

% Pacotes para plotar gráficos
\usepackage{tikz, tkz-fct}
\usepackage{pgfplots}
\pgfplotsset{width=8cm,compat=1.9}
\usepackage{pgfplots}
\pgfplotsset{width=10cm,compat=1.9}
\usepackage{tikz-3dplot}

% Pacote para incluir pdf na página
\usepackage{pdfpages}

% Pacore para definições
\newtheorem{theorem}{Definição}

% Gráficos
\usepackage{blindtext}
\usepackage{pgfplots}
\pgfplotsset{compat=1.9}
\usepackage{graphicx}

% Quadradinho$
\usepackage{amssymb}
%Pacote para módulo
\newcommand{\Mod}[1]{\ (\mathrm{mod}\ #1)}
% Definir tamanho das margens
\usepackage[inner=1.25in, outer=1in, top=1in, bottom=1in]{geometry}

% Pacotes para cabeçalho{fancyhdr}
\usepackage{fancyhdr}
\pagestyle{fancy}
\usepackage{lastpage} 
% Pacotes para tabelas que atravessam a página
\usepackage{longtable}
% Pacote para a tabela
\usepackage{float} 

% Define cabeçalho
\lhead{}
\chead{}
\rhead{Instituto Federal de Brasília}
% Definir tamanho da linha do cabeçalho
\renewcommand{\headrulewidth}{0.4pt}

\title{ROUGE: A Package for Automatic Evaluation of Summaries}
\author{De Matheus Loiola Pinto Curado Silva}
\date{}
% Reparar carácter invisível
\DeclareUnicodeCharacter{2212}{-}

\begin{document}
	
\maketitle

\section*{Introdução}

ROUGE significa \textit{Recall-Oriented Understudy for Gisting Evaluation}. Ele inclui medidas para determinar automaticamente a qualidade de um resumo ao comparar com outros resumos criados por humanos. A medição conta o número de unidades que se sobrepõem, como n-gramas, sequência de palavras e a correspondência entre pares de palavras geradas por computador.

O artigo introduz quatro diferentes tipos do ROUGE: ROUGE-N, ROUGE-L, ROUGE-W, ROUGE-S.

\subsection*{ROUGE-N}

ROUGE-N é um recall de n-grama entre o resumo candidato com os resumos referências. Essa métrica possui uma fórmula que tende muito ao recall.

Ao controlar as referências que adicionamos na métrica, é possível avaliar diferentes aspectos de resumos. Isso dá mais peso ao realizar a correspondência entre n-grams em referências múltiplas. Dessa forma, essa métrica favorece resumos que são mais similares no consenso geral.

\subsection*{ROUGE-L}

Essa métrica é baseada na LCS (Longest Common Subsequence). Ela possui os fatores de precisão e recall, e estudos demonstram que a lógica dessa métrica produziu estimativas tão boas quanto o BLEU.

Uma vantagem de usar a LCS é que não é necessário correspondências consecutivas, mas sim em sequência. A outra vantagem é automaticamente incluir a maior sequência comum de n-gramas.

Por recompensar ocorrências em sequência, ROUGE-L também captura a estrutura de nível de sentença de uma maneira natural.

Uma desvantagem é que a LCS apenas conta a principal sequência de palavras. Dessa forma, sequências menores ou outras LCSs não são refletidas no score final.

\subsection*{ROUGE-W}

ROUGE-W significa \textit{Weighted Longest Common Subsequence}. A LCS vista anteriormente possui o problema de não diferenciar LCSes de outros tamanhos ou sequências. Dessa forma, foi utilizado o algoritmo de Programação Dinâmica para lembrar do tamanho de ocorrências consecutivas em uma matriz 2D, e utilizar ela para o cálculo das métricas.

\subsection*{ROUGE-S}

Skip-bigram é qualquer par de palavras em ordem de sentença, permitindo espaços arbitrários. "Skip-bigram co-occurrence statistics" medem a sobreposição de skip-
bigrams entre a tradução candidata e o conjunto de traduções referência.

Comparando skip-bigram com a LCS, o skip-bigram conta todas as palavras correspondentes em ordem enquanto a LCS conta apenas a subsequência comum mais longa.

\subsubsection*{ROUGE-SU}

ROUGE-SU é uma extensão do ROUGE-S. A ROUGE-S tem um problema que é não dar qualquer crédito a uma sentença candidata se a sentença não tem nenhum par de palavras co-ocorrendo com suas referências. Para contornar isso, o ROUGE-S foi estendido com a adição de unigramas como uma unidade de contagem.

\section*{Conclusões}

ROUGE-2, ROUGE-L, ROUGE-W E ROUGE-S funcionaram bem em tarefas de resumo de um único documento. ROUGE-1, ROUGE-L, ROUGE-W, ROUGE-SU4 e ROUGE-SU9 performaram bem em resumos pequenos (estilo resumos de \textit{headlines}.

Em um estudo separado, ROUGE-L, ROUGE-W e ROUGE-S se mostraram muito efetivos na avaliação automática de tradução de máquina.


\end{document}