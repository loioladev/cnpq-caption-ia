% O documento possui um erro de segurança por causa do comando -shell-escape na compilacao
\documentclass[12pt]{article}
\usepackage[utf8]{inputenc}
% Colocar documento em portugues
\usepackage[brazil]{babel}
% Incluir asmmath e outras ferramentas matemátias
\usepackage{mathtools}

% Pacote para demonstração
\usepackage{amsthm}

% Pacote para cores
\usepackage{xcolor}
%\definecolor{pastelyellow}{HTML}{fdfd96} - Definir cor

% Pacote matemático
\usepackage{amsmath}

% Pacote para imagens um ao lado da outra
\usepackage{subfig}

% Pacote e função para espaçamento 1.5
\usepackage{setspace}
\onehalfspacing%

% Pacote para códigos
\usepackage{minted}

% Para representar símbolos dos conjuntos
\usepackage{amsfonts}

% Pacotes para plotar gráficos
\usepackage{tikz, tkz-fct}
\usepackage{pgfplots}
\pgfplotsset{width=8cm,compat=1.9}
\usepackage{pgfplots}
\pgfplotsset{width=10cm,compat=1.9}
\usepackage{tikz-3dplot}

% Pacote para incluir pdf na página
\usepackage{pdfpages}

% Pacore para definições
\newtheorem{theorem}{Definição}

% Gráficos
\usepackage{blindtext}
\usepackage{pgfplots}
\pgfplotsset{compat=1.9}
\usepackage{graphicx}

% Quadradinho$
\usepackage{amssymb}
%Pacote para módulo
\newcommand{\Mod}[1]{\ (\mathrm{mod}\ #1)}
% Definir tamanho das margens
\usepackage[inner=1.25in, outer=1in, top=1in, bottom=1in]{geometry}

% Pacotes para cabeçalho{fancyhdr}
\usepackage{fancyhdr}
\pagestyle{fancy}
\usepackage{lastpage} 
% Pacotes para tabelas que atravessam a página
\usepackage{longtable}
% Pacote para a tabela
\usepackage{float} 

% Define cabeçalho
\lhead{\today}
\chead{}
\rhead{Instituto Federal de Brasília}
% Definir tamanho da linha do cabeçalho
\renewcommand{\headrulewidth}{0.4pt}

\title{Resumo: BLEU: a Method for Automatic Evaluation of Machine Translation}
\author{De Matheus Loiola Pinto Curado Silva}
\date{\today}
% Reparar carácter invisível
\DeclareUnicodeCharacter{2212}{-}

\begin{document}
	
\maketitle

\section*{Introdução}

A avaliação humana nas traduções de máquina pesam em inúmeros aspectos, como em fidelidade, fluência e a adequação da frase com a original. Entretanto, esse processo pode demorar semanas ou meses para terminar, e isso é um problema pois os desenvolvedores das máquinas de tradução precisam monitorar seus resultados e suas mudanças de forma diária, para evitar podar as ideias ruins e melhorar o sistema.

Porém, como uma pessoa mede a qualidade de uma tradução? Quanto maior perto a tradução da máquina estiver de uma tradução profissional humana, melhor. Para isso, é necessário verificar o quão perto ela está da tradução original pela quantidade de palavras semelhantes entre eles.

\section*{The Baseline BLUE Metric}

Existem várias traduções perfeitas de alguma sentença, e elas podem variar na escolha de palavras.

Assim, a principal tarefa de um implementador BLEU é comparar \textit{n-grams} de traduções candidatas com \textit{n-grams} da tradução de referência e contar o número de igualdades. Quanto mais igualdades, maior é seu rank entre os candidatos.

Uma sentença candidata não deve ser nem tão longa, nem tão curta, e a validação da métrica deve verificar isso. Caso a sentença seja maior que a sentença de referência, essa sentença logo é penalizada pela métrica de precisão do \textit{n-gram}, então não é necessária penalizar de novo.

\section*{A avaliação BLEU}

A métrica BLEU varia entre 0 e 1. Poucas traduções vão atingir o escore máximo a não ser que sejam idênticos a sentença de referência.

\section*{Conclusão}

BLEU irá acelerar a área de pesquisa e desenvolvimento em máquinas de tradução. A força do BLEU é que ele se correlaciona altamente com os julgamentos humanos, fazendo a média dos erros de julgamento de sentenças individuais sobre um corpus de teste, em vez de tentar adivinhar o julgamento humano exato para cada sentença: a quantidade leva à qualidade.

\end{document}