% O documento possui um erro de segurança por causa do comando -shell-escape na compilacao
\documentclass[12pt]{article}
\usepackage[utf8]{inputenc}
% Colocar documento em portugues
\usepackage[brazil]{babel}
% Incluir asmmath e outras ferramentas matemátias
\usepackage{mathtools}

% Pacote para demonstração
\usepackage{amsthm}

% Pacote para cores
\usepackage{xcolor}
%\definecolor{pastelyellow}{HTML}{fdfd96} - Definir cor

% Pacote matemático
\usepackage{amsmath}

% Pacote para imagens um ao lado da outra
\usepackage{subfig}

% Pacote e função para espaçamento 1.5
\usepackage{setspace}
\onehalfspacing%

% Pacote para códigos
\usepackage{minted}

% Para representar símbolos dos conjuntos
\usepackage{amsfonts}

% Pacotes para plotar gráficos
\usepackage{tikz, tkz-fct}
\usepackage{pgfplots}
\pgfplotsset{width=8cm,compat=1.9}
\usepackage{pgfplots}
\pgfplotsset{width=10cm,compat=1.9}
\usepackage{tikz-3dplot}

% Pacote para incluir pdf na página
\usepackage{pdfpages}

% Pacore para definições
\newtheorem{theorem}{Definição}

% Gráficos
\usepackage{blindtext}
\usepackage{pgfplots}
\pgfplotsset{compat=1.9}
\usepackage{graphicx}

% Quadradinho$
\usepackage{amssymb}
%Pacote para módulo
\newcommand{\Mod}[1]{\ (\mathrm{mod}\ #1)}
% Definir tamanho das margens
\usepackage[inner=1.25in, outer=1in, top=1in, bottom=1in]{geometry}

% Pacotes para cabeçalho{fancyhdr}
\usepackage{fancyhdr}
\pagestyle{fancy}
\usepackage{lastpage} 
% Pacotes para tabelas que atravessam a página
\usepackage{longtable}
% Pacote para a tabela
\usepackage{float} 

% Define cabeçalho
\lhead{\today}
\chead{}
\rhead{Instituto Federal de Brasília}
% Definir tamanho da linha do cabeçalho
\renewcommand{\headrulewidth}{0.4pt}

\title{Resumo: Microsoft Coco - Common objects in context}
\author{De Matheus Loiola Pinto Curado Silva}
\date{\today}
% Reparar carácter invisível
\DeclareUnicodeCharacter{2212}{-}

\begin{document}
	
\maketitle

\tableofcontents

\section*{Introdução}

Compreender cenas é um processo que possui inúmeras etapas, visto que é necessário compreender os atributos da cena, a relação entre eles, e o desenvolvimento de uma semântica para descrevê-los. Um exemplo que ajudou nessa compreensão é o \textit{dataset ImageNet}, que possibilitou inúmeras descobertas nas áreas de classificação e detecção de objetos. 

Assim, o artigo introduz um novo \textit{dataset}, que é focado na \textbf{detecção visual de objetos, contextualização entre objetos e a localização precisa deles em imagens não icônicas}, para que seja possível avançar nas pesquisas de detecção de contexto e cenas, sem focar em apenas objetos isolados (imagens icônicas).

Dessa forma, para atingir esse objetivo, foram agrupadas imagens contendo relações contextuais e visualizações não icônicas, ou seja, os objetos estão fora do foco na imagem. O \textit{dataset} possui 91 categorias com 82 delas contendo mais do que 5000 imagens. A ideia do \textit{dataset} também é ter menos categorias, mas ter mais instâncias por categoria.

\section*{Trabalho relacionado}

Os \textit{datasets} relacionados com detecção de objetos podem ser divididos em três grupos: aqueles que se referem a classificação de objetos, detecção de objetos ou rotulagem semântica da cena.

\begin{itemize}
    \item
    \textbf{Classificação de objetos}: A tarefa de classificação de objetos requer rótulos binários indicando se os objetos são presente em uma imagem. O \textit{dataset} utilizado para isso pode ser o \textbf{ImageNet}, contendo mais de 22.000 categorias de objetos.

    \item
    \textbf{Detecção de objetos}: Detectar um objeto envolve tanto afirmar que um objeto pertencente a uma classe especificada e se ele está presente na imagem. O \textit{dataset} utilizado para isso pode ser o \textbf{PASCAL VOC}.

    \item
    \textbf{Rotulagem semântica de cenas}: A tarefa de rotular objetos semânticos em uma cena requer que cada pixel de uma cena imagem seja rotulada como pertencente a uma categoria, como céu, cadeira, chão... O \textit{dataset} utilizado para isso pode ser o \textbf{SUN}.
\end{itemize}

\section*{Coleção de imagens}

O \textit{dataset} tem o foco em objetos, ou seja, pessoas, carros, cadeiras, etc. e não possui foco em coisas que geralmente não tem limites, como ruas, grama ou céu. Além disso, as categorias são limitadas a termos básicos, frequentemente usados por humanos, como utilizar a categoria "cachorro" ao invés de "pastor-alemão" ou outras raças.

\section*{Anotações de imagens}

Essa seção demonstra o processo para criar o \textit{dataset MS-COCO}.

A primeira etapa foi definir quais categorias estavam presentes em cada imagem. Após isso, todas as instâncias dos objetos de certa categoria são rotuladas. Dessa forma, o estagio final é segmentar cada instância do objeto na imagem. Apos essas três etapas, os pesquisadores analisam os resultados e verificaram quais imagens precisariam de revisão. Por fim, foram adicionadas cinco legendas para cada imagem no \textit{dataset}.

\section*{Estatísticas e resultados finais}

O \textit{MS COCO} foi desenvolvido para detecção e segmentação de objetos em seu contexto natural. O \textit{dataset} foi separado em 50$\%$ das imagens para treinamento, 25$\%$ para validação e teste.

É notável, também, que em comparação com o \textit{PASCAL VOC}, existe uma diferença de performance entre o \textit{dataset MS COCO} ao treinar uma rede neural, visto que este último possui imagens mais difíceis (não icônicas) nas quais os objetos estão parcialmente ocultos.

\end{document}