% O documento possui um erro de segurança por causa do comando -shell-escape na compilacao
\documentclass[12pt]{article}
\usepackage[utf8]{inputenc}
% Colocar documento em portugues
\usepackage[brazil]{babel}
% Incluir asmmath e outras ferramentas matemátias
\usepackage{mathtools}

% Pacote para demonstração
\usepackage{amsthm}

% Pacote para cores
\usepackage{xcolor}
%\definecolor{pastelyellow}{HTML}{fdfd96} - Definir cor

% Pacote matemático
\usepackage{amsmath}

% Pacote para imagens um ao lado da outra
\usepackage{subfig}

% Pacote e função para espaçamento 1.5
\usepackage{setspace}
\onehalfspacing%

% Pacote para códigos
\usepackage{minted}

% Para representar símbolos dos conjuntos
\usepackage{amsfonts}

% Pacotes para plotar gráficos
\usepackage{tikz, tkz-fct}
\usepackage{pgfplots}
\pgfplotsset{width=8cm,compat=1.9}
\usepackage{pgfplots}
\pgfplotsset{width=10cm,compat=1.9}
\usepackage{tikz-3dplot}

% Pacote para incluir pdf na página
\usepackage{pdfpages}

% Pacore para definições
\newtheorem{theorem}{Definição}

% Gráficos
\usepackage{blindtext}
\usepackage{pgfplots}
\pgfplotsset{compat=1.9}
\usepackage{graphicx}

% Quadradinho$
\usepackage{amssymb}
%Pacote para módulo
\newcommand{\Mod}[1]{\ (\mathrm{mod}\ #1)}
% Definir tamanho das margens
\usepackage[inner=1.25in, outer=1in, top=1in, bottom=1in]{geometry}

% Pacotes para cabeçalho{fancyhdr}
\usepackage{fancyhdr}
\pagestyle{fancy}
\usepackage{lastpage} 
% Pacotes para tabelas que atravessam a página
\usepackage{longtable}
% Pacote para a tabela
\usepackage{float} 

% Define cabeçalho
\lhead{\today}
\chead{}
\rhead{Instituto Federal de Brasília}
% Definir tamanho da linha do cabeçalho
\renewcommand{\headrulewidth}{0.4pt}

\title{Resumo: SPICE -  Semantic Propositional Image Caption Evaluation}
\author{De Matheus Loiola Pinto Curado Silva}
\date{\today}
% Reparar carácter invisível
\DeclareUnicodeCharacter{2212}{-}

\begin{document}
	
\maketitle

\section*{Introdução}

O artigo diz que métricas automáticas de avaliação são primariamente sensíveis à sobreposição de n-gramas, o que não é necessário nem suficiente para tarefas de simulação de julgamento humano. Assim, o autor hipotetiza que o conteúdo da semântica proposicional é um componente importante para a avaliação da legenda humana, e propõe uma nova métrica automática de avaliação de legendas chamado SPICE, \textit{Semantic Propositional Image Caption Evaluation}.

Avaliações abrangentes em uma variedade de modelos e conjuntos de dados indicam que o SPICE captura julgamentos humanos sobre legendas geradas por modelos melhores do que outras métricas automáticas. Além disso, o SPICE pode responder perguntas como "qual gerador de legendas entende melhor as cores?" e "Os geradores de legendas podem contar?"

No artigo, apresentam essa nova métrica de avaliação automática de legendas de imagens que mede a qualidade das legendas geradas através da análise de seu conteúdo semântico. O método se assemelha muito ao julgamento humano, oferecendo a vantagem adicional de que o desempenho de qualquer modelo pode ser analisado com mais detalhes do que com outros métricas automatizadas.

Eles estimamos a qualidade da legenda por transformar legendas candidatas e de referência em um representante semântico baseado em grafos e chamado de "scene graphs". Esse "scene graph" explicitamente encoda os objetos, atributos e relações achados nas legendas da imagem, abstraindo a maior parte lexicográfica e sintática da linguagem natural no processo.

\section*{Métrica SPICE}

Dado uma legenda candidata $c$ e um conjunto de legendas referência $S$ associadas a imagem, o objetivo é computar uma pontuação que captura a similaridade entre $C$ e $S$. Como o algoritmo explora a estrutura semântica das descrições da cena, ela dá preferência a substantivos, e por isso é melhor em avaliar legendas geradas por computador. O SPICE foca exclusivamente no significado semântico.

Ele utiliza o F-score, assim, sua pontuação está limitada entre 0 e 1. O SPICE mede o quão bem os geradores de legendas recuperam objetos, atributos e as relações entre eles. Uma preocupação potencial, então, é que a métrica poderia ser 'viciada', gerando legendas que representam apenas objetos, atributos e relações, ignorando outros aspectos importantes da gramática e da sintaxe. Como o SPICE negligencia a fluência, como acontece com as métricas de n-gramas, ele assume que as legendas são bem formadas.

\section*{Avaliação da métrica}

em testes realizados em uma competição do MS COCO, apenas o SPICE recompensou legendas detalhadas, enquanto outras métricas penalizaram essas legendas, como visto abaixo:

\begin{figure}[H]
    \centering
    \includegraphics{avaliacao1.png}
\end{figure}

Em relação a percepção de cores e contagem, a métrica conseguiu avaliar bem a cor, quantidade e tamanho dos atributos nas legendas. Além disso, utilizando a correlação de Kendall's em nível de legenda, a métrica não se saiu muito diferentes das outras métricas criadas, porém, SPICE aproxima o julgamento humano melhor quando agregado com outras legendas.

\end{document}