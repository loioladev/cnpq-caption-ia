% O documento possui um erro de segurança por causa do comando -shell-escape na compilacao
\documentclass[12pt]{article}
\usepackage[utf8]{inputenc}
% Colocar documento em portugues
\usepackage[brazil]{babel}
% Incluir asmmath e outras ferramentas matemátias
\usepackage{mathtools}

% Pacote para demonstração
\usepackage{amsthm}

% Pacote para cores
\usepackage{xcolor}
%\definecolor{pastelyellow}{HTML}{fdfd96} - Definir cor

% Pacote matemático
\usepackage{amsmath}

% Pacote para imagens um ao lado da outra
\usepackage{subfig}

% Pacote e função para espaçamento 1.5
\usepackage{setspace}
\onehalfspacing%

% Pacote para códigos
\usepackage{minted}

% Para representar símbolos dos conjuntos
\usepackage{amsfonts}

% Pacotes para plotar gráficos
\usepackage{tikz, tkz-fct}
\usepackage{pgfplots}
\pgfplotsset{width=8cm,compat=1.9}
\usepackage{pgfplots}
\pgfplotsset{width=10cm,compat=1.9}
\usepackage{tikz-3dplot}

% Pacote para incluir pdf na página
\usepackage{pdfpages}

% Pacore para definições
\newtheorem{theorem}{Definição}

% Gráficos
\usepackage{blindtext}
\usepackage{pgfplots}
\pgfplotsset{compat=1.9}
\usepackage{graphicx}

% Quadradinho$
\usepackage{amssymb}
%Pacote para módulo
\newcommand{\Mod}[1]{\ (\mathrm{mod}\ #1)}
% Definir tamanho das margens
\usepackage[inner=1.25in, outer=1in, top=1in, bottom=1in]{geometry}

% Pacotes para cabeçalho{fancyhdr}
\usepackage{fancyhdr}
\pagestyle{fancy}
\usepackage{lastpage} 
% Pacotes para tabelas que atravessam a página
\usepackage{longtable}
% Pacote para a tabela
\usepackage{float} 

% Define cabeçalho
\lhead{}
\chead{}
\rhead{Instituto Federal de Brasília}
% Definir tamanho da linha do cabeçalho
\renewcommand{\headrulewidth}{0.4pt}

\title{METEOR: An Automatic Metric for MT Evaluation with Improved Correlation with Human Judgments}
\author{De Matheus Loiola Pinto Curado Silva}
\date{}
% Reparar carácter invisível
\DeclareUnicodeCharacter{2212}{-}

\begin{document}
	
\maketitle

\section*{Introdução}

O artigo descreve METEOR, uma métrica automática para avaliação de tradução de máquina que é baseada em um conceito generalizado de casamento de unigramas entre traduções de máquina e traduções de humano. 

A métrica precisa ser consistente, confiável e generalizada. Dessa forma, METEOR foi desenvolvida para consertar inúmeras fraquezas na métrica BLEU. Essa métrica é baseada em um casamento de palavra por palavra e uma ou mais traduções referência. Essa métrica não só faz o casamento de palavras idênticas, mas como também faz o casamento entre palavras que possuem o mesmo sinônimo ou variantes dessa palavra.

Cada casamento é dado uma pontuação de acordo a precisão e recall de unigramas e o quanto as palavras estão fora de ordem na tradução da máquina. Foi verificado, também, que o recall é mais importante do que a precisão na validação das traduções.

\section*{A métrica METEOR}

O principal problema na métrica BLEU é a medição de unigramas que se sobrepôem (palavras únicas) e palavras com n-grams de alta ordem. O principal componente do BLEU é a precisão, calculada a partir dos n-grams casados com os n-grams da tradução, e não leva o recall em conta diretamente, e compensa isso utilizando uma penalidade no tamanho da tradução (Brevity Penalty).

Os problemas citados pelo artigo, no geral, são:
\begin{itemize}
    \item Falta do recall.
    \item Uso de uma ordem alta de N-grams.
    \item Falta de casamento de palavras explícito entre a tradução e a referência.
    \item Uso de média geometrica dos N-grams.
\end{itemize}

Assim, METEOR foi desenvolvido para tratar desses problemas. Ele avalia uma tradução calculando uma pontuação baseada em correspondências explícitas palavra a palavra entre a tradução e uma tradução de referência.

METEOR cria um alinhamento entre duas strings. Um alinhamento é como um mapeamento entre unigramas, de modo que cada unigrama de cada string é mapeado para zero ou um unigrama na outra string. Esse alinhamento é produzido incrementalmente através de uma série de etapas.

\section*{Avaliação da métrica METEOR}

A comparação em nível de sistema é a seguinte:

\begin{figure}[H]
    \centering
    \includegraphics{compracao_meteor.png}
    \label{fig:enter-label}
\end{figure}

É observado que o recall, por si só, se correlaciona com avaliação humana muito melhor do que precisão, e que a combinação dos dois usando a fórmula Fmean resulta em melhorias adicionais.

\end{document}